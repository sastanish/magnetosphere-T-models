\documentclass{article}

\usepackage{graphicx}
\usepackage{amsmath,amssymb}

\graphicspath{ {./} }

\begin{document}
\section{Good Cases}
\subsection{The SymH metric is a good predictor of near-earth x-points}

We have used empirical models to approximate the magneto-tail's near earth field during strong disturbances. During these strong storms, these models show a strong magnetic flux rope forms between the magneto-tail current sheet and the near-earth dipolar field. The presense of this flux rope near earth, causes a magnetic configureation that is consistent with traditional x-point reconnection. We find that the nearest reconection to the earth's surface occurs at the minima of the SymH index.
\begin{center}
\includegraphics[width=0.45\linewidth]{s01/x-point_location.png}
\hspace{0.01\linewidth}
\includegraphics[width=0.45\linewidth]{s06/x-point_location.png}
\end{center}

\begin{center}
\includegraphics[width=0.45\linewidth]{s01/omni_data.png}
\hspace{0.01\linewidth}
\includegraphics[width=0.45\linewidth]{s06/omni_data.png}
\end{center}

\subsection{An indicator of this near-earth reconnection is the enhancement in ground showers seen in neutron monitor data.}

\begin{center}
\includegraphics[width=0.45\linewidth]{s01/neutron_monitor_data.png}
\hspace{0.01\linewidth}
\includegraphics[width=0.45\linewidth]{s06/neutron_monitor_data.png}
\end{center}


\subsection{Strong reconnection in the tail can cause the ejection of plasmoids that results in near earth reconnection.}
\subsubsection{The position of this reconnection is determined by a matching of the tail pressure with the ambient pressure of the earth's magnetosphere.}

\begin{center}
\includegraphics[width=0.45\linewidth]{s01/pressure.png}
\hspace{0.01\linewidth}
\includegraphics[width=0.45\linewidth]{s06/pressure.png}
\end{center}

\section{Uncertain cases}

\subsection{March 2022}
\begin{center}
\includegraphics[width=0.45\linewidth]{s02/x-point_location.png}
\hspace{0.01\linewidth}
\includegraphics[width=0.45\linewidth]{s02/omni_data.png}

\includegraphics[width=0.45\linewidth]{s02/neutron_monitor_data.png}
\hspace{0.01\linewidth}
\includegraphics[width=0.45\linewidth]{s02/pressure.png}
\end{center}

\end{document}
